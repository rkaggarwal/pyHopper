\documentclass[12pt]{article}
\usepackage{amsfonts}
\usepackage{booktabs}
\usepackage[margin=1in]{geometry}
\usepackage{amsmath}
\usepackage{gensymb}
\usepackage{cite}
\usepackage{sectsty}
\usepackage{graphicx}
\usepackage{array}
\usepackage{caption}
\usepackage{hyperref}
\usepackage{subfig}
\usepackage{float}
\usepackage{rotating}
\usepackage{tikz}
\usepackage{fixltx2e}
\usepackage{enumitem}
\usepackage{setspace}
\usepackage[normalem]{ulem}
\newcommand{\msout}[1]{\text{\sout{\ensuremath{#1}}}}
\newcommand\numberthis{\addtocounter{equation}{1}\tag{\theequation}}

\newenvironment{conditions}
  {\par\vspace{\abovedisplayskip}\noindent\begin{tabular}{>{$}l<{$} @{${}={}$} l}}
  {\end{tabular}\par\vspace{\belowdisplayskip}}

\setlength\parindent{0pt}
\usepackage{gensymb}
\begin{document}

\section*{pyHopper Dynamics}
Description: \textit{{2D Thrust-Vectored Rocket Hopper, with the engine thrust offset from vehicle CG.  This allows for 3 DOF in positional controllability with only 2 DOF in control (thrust and angle).}}

\bigskip
\normalsize
The equations of motion for the system are simply derived from Newton's laws:

\begin{align*}
	f_4(z, u) = \ddot{x} &= -\frac{F_{t} \sin(\theta + \phi)}{m} \\
	f_5(z, u) = \ddot{y} &= \frac{F_{t} \cos(\theta + \phi) - mg}{m}\\
	f_6(z, u) = \ddot{\theta} &= -\frac{F_{t} \,d \sin(\phi)}{J},
\end{align*}

where

\begin{conditions}
m & mass of hopper [kg] \\
J & rotational inertia of hopper [kg-m$^2$] \\
F_{t} & engine thrust [N] \\
\theta & hopper heading angle, off vertical [rad] \\
\phi & thrust vector angle, off hopper longitudinal centerline [rad] \\
g & gravitational acceleration [m/s$^2$] \\
d & distance from thrust vector to hopper center of gravity (i.e. thrust moment arm) \,[m]. \\
\end{conditions}

\bigskip  A state-space formulation is more amenable to control design; but, strong non-linearities in the equations of motion don't allow for a simple determination of the A/B/C/D matrices.

\bigskip However, because the hopper is expected to remain upright during initial trajectory tracking (e.g. point-to-point movement), linearizing about an upright, zero-velocity ``hovering" configuration is reasonable and gives us a usable state space formulation.

\bigskip Viewing the control efforts as deltas off equilibrium (trim), the following subset of the state-space defines equilibrium configurations where linearization is acceptable:

\begin{align*}
      {x}_{eq} &\subset \mathbb{R} \\
      {y}_{eq} &\subset  \mathbb{R}^+ \\
      {\theta}_{eq} &= 0 \\
	\dot{x}_{eq} &= 0 \\
	\dot{y}_{eq} &= 0 \\
	\dot{\theta}_{eq} &= 0 \\
	F_{t, eq} &= mg \\
	\phi_{eq} &= 0 \\
\end{align*}


\bigskip Then, the gradients of the Newtonian dynamics are calculated in this equilibrium subspace to give a linearized state-space representation:

\begin{align*}
A &= \begin{pmatrix}
0 & 0 & 0 & 1 & 0 & 0 \\
0 & 0 & 0 & 0 & 1 & 0 \\
0 & 0 & 0 & 0 & 0 & 1 \\
\frac{\partial f_4}{\partial x}  & \frac{\partial f_4}{\partial y}  & \frac{\partial f_4}{\partial \theta}  & \frac{\partial f_4}{\partial \dot{x}}  & \frac{\partial f_4}{\partial \dot{y}}  & \frac{\partial f_4}{\partial \dot{\theta}}  \\
\frac{\partial f_5}{\partial x}  & \frac{\partial f_5}{\partial y}  & \frac{\partial f_5}{\partial \theta}  & \frac{\partial f_5}{\partial \dot{x}}  & \frac{\partial f_5}{\partial \dot{y}}  & \frac{\partial f_5}{\partial \dot{\theta}}  \\
\frac{\partial f_6}{\partial x}  & \frac{\partial f_6}{\partial y}  & \frac{\partial f_6}{\partial \theta}  & \frac{\partial f_6}{\partial \dot{x}}  & \frac{\partial f_6}{\partial \dot{y}}  & \frac{\partial f_6}{\partial \dot{\theta}}  \\
\end{pmatrix} \\ \\
B &= \begin{pmatrix}
0 & 0 \\
0 & 0 \\
0 & 0 \\
\frac{\partial f_4}{\partial F_t}  & \frac{\partial f_4}{\partial \phi} \\
\frac{\partial f_5}{\partial F_t}  & \frac{\partial f_5}{\partial \phi} \\
\frac{\partial f_6}{\partial F_t}  & \frac{\partial f_6}{\partial \phi}
\end{pmatrix}, \\
\end{align*}

where 
\begin{align*}
z &= \begin{pmatrix}
x \\
y \\
\theta \\
\dot{x} \\
\dot{y} \\
\dot{\theta} \\
\end{pmatrix} \\ \\
u &= \begin{pmatrix}
\delta F_t \\
\delta\phi \\
\end{pmatrix}
\end{align*}

and

\begin{equation*}
\dot{z} = Az + Bu.
\end{equation*}

Evaluating the gradients gives the A and B matrices explicitly:

\begin{align*}
A &= \begin{pmatrix}
0 & 0 & 0 & 1 & 0 & 0 \\
0 & 0 & 0 & 0 & 1 & 0 \\
0 & 0 & 0 & 0 & 0 & 1 \\
0 & 0 & -g & 0 & 0 & 0 \\
0 & 0 & 0 & 0 & 0 & 0 \\
0 & 0 & 0 & 0 & 0 & 0 \\
\end{pmatrix} \\ \\
B &= \begin{pmatrix}
0 & 0 \\
0 & 0 \\
0 & 0 \\
0 & -g \\
1 & 0 \\
0 & -\frac{mgd}{J} \\
\end{pmatrix}. \\
\end{align*}

\bigskip This formulation is used in the LQR and other optimal control design.  Simulation is written in Python with help from the Control Systems Library (https://pypi.org/project/control/).






\end{document}

